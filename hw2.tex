\documentclass{article}
\usepackage[utf8]{inputenc}

% ECS 261 HW2 Problem Set

\usepackage{amsmath}
\usepackage{amssymb}
\usepackage{amsthm}
\usepackage{enumerate}

\usepackage[margin=1.5in]{geometry}

\usepackage{hyperref}

\begin{document}

\title{ECS 261 HW2 Problem Set}
\author{Your Name}
\date{Due Friday, May 2, 2025}

\maketitle

\begin{enumerate}
\item
Say that a specification is ``expressible in Z3'' if it can be written as a Python program \texttt{def spec(prog)} that takes as input a string containing the source code of a program \texttt{prog}, and returns a Z3 formula.
The specification is true exactly when the Z3 formula is valid.
Show that for any two different programs \texttt{prog1} and \texttt{prog2},
they are distinguishable: there exists a spec which \texttt{prog1} satisfies and \texttt{prog2} does not.
Prove your answer.

\item
Consider the theory of integers and real numbers that we saw in class.
The syntax uses function symbols $\{+, *, -\}$ and relation symbols $\{=, <\}$:
\begin{verbatim}
    Var ::= n1, n2, n3, ...
    Expr ::= Expr + Expr | Expr - Expr | Expr * Expr
            | Var | 0 | 1
    Formula ::= Formula v Formula | Formula ^ Formula | !Formula
            | Expr == Expr | Expr < Expr
\end{verbatim}

\begin{enumerate}[(a)]
\item Give an example of a formula that is satisfiable over the real numbers but not the integers.
\item Prove formally that every expression satisfiable over the integers is satisfiable over the real numbers.
\item Define the theory of truncated 8-bit integers as follows: again we use the same grammar, but now integers are forced to be between $-2^{32}$ and $2^{32} - 1$, inclusive. If an expression goes out-of-bounds, we wrap around to the maximum value: for example, $2^{30} + 2^{30} = 2^{31} - 1$ as this is the maximum value available. Similarly $2^{16} * 2^{30} = 2^{31} - 1$, $-5 - (2^{31} - 1) = -2^{31}$ and $(-1) * (-2^{32}) = 2^{31} - 1$.

Is every formula satisfiable over the integers satisfiable over this theory?
Is every formula satisfiable over this theory satisfiable over the integers?
Justify both answers.
\end{enumerate}

\item
\emph{Symbolic execution} is a static analysis technique where we ``run'' a program by evaluating expressions as symbols and formulas,
instead of using concrete values.
For example, the function in Python \texttt{def add(x):} with body \texttt{return x + 7} would normally be evaluated
by plugging an integer like $10$ and returning $17$. With symbolic execution, instead we plug in the value
\texttt{z3.Int("x")} and return the value \texttt{z3.Int("x") + 7}, that is we return the symbolic integer expression $(x + 7)$. We can then answer questions aout the behavior of the function, for example, ``does there exist an input for which the output is exactly $10$?'' or ``does there exist an input for which the output is out of bounds for an array?'' by making a query to Z3.

Use Z3 to decide whether the following has an infinite loop bug.
The program has 4 paths leading up to the potential infinite loop; you don't need to implement a general symbolic execution engine, just describe the Z3 formula along each path.

\begin{verbatim}
TODO
\end{verbatim}

Report your answer:

\begin{enumerate}[(a)]
\item Did Z3 find an infinite loop bug, prove that no such bug exists, or was it inconclusive (UNKNOWN on one or more paths)?
\item How long did you find it took Z3 to solve all paths?
\item Suppose you wanted to include all paths in the same Z3 formula to reduce the number of queries to Z3.
(This can be helpful for performance in some cases.) How would you propose doing so with the above technique?
Sketch your answer by providing a single Z3 formula that would result from this solution.
\end{enumerate}

Attach your Z3 code with your submission as a file \texttt{hw2\_q2.py}.

\end{enumerate}

\end{document}
